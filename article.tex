\thispagestyle{empty}
{\theme{A comparative study of automated CT-based 3D finite element modeling approaches and refinement techniques for biomechanical analysis.}}

\vspace{5mm}

\begin{center} Eduard B. Demishkevich$^{1,2}$, Sergey S. Gavriushin$^{1,2}$, Kirill K. Kazakhmedov $^{1}$, Vladimir P. Shilyaev$^{1}$  \\

\vspace{5mm}

{\small $^{1}$ Bauman Moscow State Technical University, 5c1, 2nd Baumanskaya st., 105005, Moscow, Russian Federation \\
$^{2}$ Mechanical Engineering Research Institute of the Russian Academy of Sciences, 4, Malyi Kharitonievsky pereulok, 101990, Moscow, Russian Federation \\
mail@edtech.ru }
\end{center}

\begin{changemargin}{0.5cm}{0.5cm}
{\small \textbf{Abstract.} The objective of this study was to compare different techniques to build finite element models for biomechanical analysis
basing on computer tomography layerwise images. The first approach is the simplest method consisting in assignment of cubic finite element
model to each voxel of the segmented cloud of points obtained from CT-slices. The second approach is based on application of the marching
cubes method and building a tetrahedral Delaunay mesh. Both methods were used to perform a biomechanical simulation of human's lumbar vertebra and upper jaw.
Mesh quality and simulation results compared in the study. The specifics of each technique described.}

{\small \textbf{Keywords:} finite-elements method, biomechanics}

\end{changemargin}

\ctheme{1. Introduction}

% описать современный подход к индивидуализированному лечению: по томограммам строится МКЭ модель, затем производится биомеханический расчёт,
% чтобы построить оптимальный план лечения или изготовить кастомизированный хирургический инструмент

% описать подходы к построению 3D FEM моделей

% сослаться на предыдущие работы и указать, что в них не использовались техники улучшения моделей, которые могут оказывать значительное влияние на точность
% расчетов

\ctheme{2. Voxel-based modelling}

\ctheme{3. Geometry-based modelling}

\ctheme{4. Simulation results}

\ctheme{5. Conclusion}

\ctheme{References}